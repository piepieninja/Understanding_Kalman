\title{\textbf{Understanding the Kalman Filter}\\
      \small{A Simple, Verbose, 1D Guide}
      }
\author{
        Caleb Adams \\
                UGA Small Satellite Research Laboratory\\
}
% \date{\today}

% https://oeis.org/wiki/List_of_LaTeX_mathematical_symbols

\documentclass[11pt]{article}

\usepackage{graphicx}
\usepackage{tikz}
\usepackage[margin=1.5cm]{geometry}
\usepackage[most]{tcolorbox}
\usepackage{chessboard}
\usepackage{float}
\usepackage{hyperref}
%\usepackage{fontspec}
\usepackage{siunitx}
\usepackage{amsfonts}
\usepackage{amsmath}
\usepackage{amssymb}

\newcommand*\circled[1]{\tikz[baseline=(char.base)]{
            \node[shape=circle,draw,inner sep=2pt] (char) {#1};}}

\storechessboardstyle{4x4}{maxfield=d4}
\usetikzlibrary{arrows,automata}
\graphicspath{ {./img/} }

\newtcblisting{commandshell}{colback=black,colupper=white,colframe=black!75!black,
listing only,listing options={language=sh},
every listing line={\textcolor{green}{\small\ttfamily\bfseries Caleb Adams \$> }}}

\begin{document}
\maketitle

The use of the Kalman Filter Algorithm is common in robotics as a control
method \cite{probrobtics}. More importantly, it can be used to control spacecraft - and
who doesn't love a good spaceship? Preferably one with an invisibility feild... but
we won't be discussing spaceships, or invisibility feilds. For this case we will
imagine a humble cart on a track. The cart can control its movment by applying a force.
The cart will also have the ability to detect it's distance along the track using
a lazer.

\section{A Motion Model}

It is simple to understand a linear motion model. For every time step, let's call
the time step $\Delta t$, we move some distance, let's call the distance $\Delta x$,
across the track. We will imagine that this track is the $x$ axis. The amount of
distance ($\Delta x$) that the cart moves in a certain amount of time ($\Delta t$)
is its velocity. We will say the the robots instantaneous change in position, the
derivative that results in velocity, is represented as $\dot{x}$. If we remember the
form definition of the derivative \cite{calculus} we see:

\[
\dot{x} = x(a) = \lim_{h\to0} \frac{f(a + h) - f(a) }{h}
\]

We can see that any movement that our cart makes along our track is given by some
change in time multiplied by the instantaneous change in position over that time,
so $\Delta x =  \dot{x} \Delta t$. If we recall from the great teacher Newton, we can
write an equation of motion \cite{Principia} as follows:

\[
x = x_0 + \dot{x} \Delta t + \frac{1}{2} \ddot{x} (\Delta t)^2
\]

Here the $x_0$ value represents the starting position of the cart along the track.
In the simplest cast, the starting position can be $0$. We can also see that the
term $\ddot{x}$ is the acceleration term $a$. I will be helpful to again remember
$F= ma$. We can rearrange this to see $F = ma \rightarrow F = m \ddot{x} \rightarrow \ddot{x} = \frac{F}{m}$,
thus:

\[
x = x_0 + \dot{x} \Delta t + \frac{1}{2} \frac{F}{m} (\Delta t)^2
\]

We can then view the velocity in the same way. With the term $\dot{x}_0$ the intial
velocity of the system:

\[
\dot{x} = \dot{x}_0 + \frac{F}{m} \Delta t
\]

Next, we want to view the system in a more compact way. To do this we see that the
cart has a state $s$ at any given time $t$. In otherwords, at any given time
the state of the cart can be discribed by its position and velocity. We can represent
this state with the existing funcitons for $x$ and $\dot{x}$:

\[
s =
\begin{bmatrix}
  x\\
  \dot{x}
\end{bmatrix}
\]

And if we continue to expand this equation we get...

\[
s =
\begin{bmatrix}
  x\\
  \dot{x}
\end{bmatrix}
=
\begin{bmatrix}
  x_0 + \dot{x} \Delta t + \frac{1}{2} \frac{F}{m} (\Delta t)^2\\
  \dot{x}_0 + \frac{F}{m} \Delta t
\end{bmatrix}
\]

Now We apply some basic linear algebra \cite{strang09}:

\[
s
=
\begin{bmatrix}
  x_0 + \dot{x} \Delta t + \frac{1}{2} \frac{F}{m} (\Delta t)^2\\
  \dot{x}_0 + \frac{F}{m} \Delta t
\end{bmatrix}
\rightarrow
s
=
\begin{bmatrix}
  x_0 + \dot{x} \Delta t\\
  \dot{x}_0
\end{bmatrix}
+
\begin{bmatrix}
  \frac{1}{2} \frac{F}{m} (\Delta t)^2\\
   \frac{F}{m} \Delta t
\end{bmatrix}
\]

\[
\rightarrow
s
=
\begin{bmatrix}
  x_0 + \dot{x} \Delta t\\
  \dot{x}_0
\end{bmatrix}
+
\begin{bmatrix}
  \frac{1}{2m} (\Delta t)^2\\
   \frac{1}{m} \Delta t
\end{bmatrix}
F
\rightarrow
s =
\begin{bmatrix}
  x_0 + \dot{x} \Delta t\\
  \dot{x}_0
\end{bmatrix}
+
\begin{bmatrix}
  \frac{(\Delta t)^2}{2m}\\
   \frac{\Delta t}{m}
\end{bmatrix}
F
\]

\[
\rightarrow
s =
\begin{bmatrix}
  1 & \Delta{t}\\
  0 & 1
\end{bmatrix}
\begin{bmatrix}
  x_0\\
  \dot{x}_0
\end{bmatrix}
+
\begin{bmatrix}
  \frac{(\Delta t)^2}{2m}\\
   \frac{\Delta t}{m}
\end{bmatrix}
F
\]


Thus,

\[
s
=
\begin{bmatrix}
  1 & \Delta t\\
  0 & 1
\end{bmatrix}
\begin{bmatrix}
  x_0\\
  \dot{x}_0
\end{bmatrix}
+
\begin{bmatrix}
  \frac{(\Delta t)^2}{2m}\\
  \frac{\Delta t}{m}
\end{bmatrix}
F
\]

It is useful to think of our system in this way. It allows us to have a single force $F$
acting upon the cart and allows us to derive its state given an initial $x_0$ and $\dot{x}_0$.

\bibliographystyle{IEEEtran}
\bibliography{sources}

\end{document}



















% yee
